\documentclass[11pt,a4paper]{article}

\usepackage{tikz}
\usepackage[parfill]{parskip}
\usepackage{amsmath}
\usepackage{amsfonts}
\usepackage{placeins}
\usepackage{cleveref}
\usepackage{bm}

\newcommand{\norm}[1]{\left\lVert#1\right\rVert}
\newcommand{\p}[1]{\left(#1\right)}

\begin{document}

\usetikzlibrary{math}

\section{Introduction}
This document describes how we can generate and transform
various shapes for rendering on a GPU. The output of each
shape is a list of vertices and some parameters to tweak
how the shape is rendered.

\section{Transformations}
There are two main ways to transform a shape. We can transform
the underlying data that describes the shape, or we can transform
the shape itself. All shapes discussed here are described by a set
of 2D points. When we transform the underlying data, it is these
points we refer to. Consider a point $\mathbf{x} \in \mathbb{R}^2$.
Any linear transformation can be described by
%
\begin{align}
  \mathbf{y} = \mathbf{A} \mathbf{x} + \mathbf{b}
\end{align}
%
If we apply this to all points before rendering, we have
transformed the underlying data. To transform the shape
we instead apply it to the points defining the triangles
that are rendered by the GPU. Of course, both approaches
can be combined.

\subsection{Homogenous Coordinates}
It is often simpler to work in homogenous coordinates,
since chaining transformations can be done by multiplying
matrices. The point $\mathbf{x}$ is represented as
%
\begin{align}
  \mathbf{x}' = \begin{bmatrix} \mathbf{x} \\ 1 \end{bmatrix}
\end{align}
%
Define a matrix
%
\begin{align}
  \mathbf{A}' = \begin{bmatrix}
    \mathbf{A} & \mathbf{b} \\
    \mathbf{0}^T & 1
    \end{bmatrix}
\end{align}
%
Then
%
\begin{align}
  \mathbf{y}' = \mathbf{A}' \mathbf{x}'
\end{align}
%
where $\mathbf{y}'$ is also a homogenous coordinate. Hence, transformations
are easily chained.

\section{Lines}
\subsection{Single Line}
Let $\mathbf{x}_1$ be one side of the line and $\mathbf{x}_2$ be the
other.  Let $w$ be desired with of the line. \Cref{fig:single-line}
shows an example of a single drawn line. To draw this line on the GPU
we need to pass the points $a$, $b$, $c$, and $d$ to a vertex shader
somehow.  It is important that both the transformation matrix
$\mathbf{A}'$ and the line width $w$, is \emph{only} used in the GPU.
With this constraint, the vertices need only be computed once.
%
\begin{figure}
  \centering
  \begin{tikzpicture}
    \tikzmath{
      \x1 = 0;
      \y1 = 0;
      \x2 = 4;
      \y2 = 2;
      \l = sqrt((\x2-\x1)^2 + (\y2-\y1)^2);
      \w = 0.4;
      \dx = \w * (\x2 - \x1) / \l;
      \dy = \w * (\y2 - \y1) / \l;
    }

    \coordinate (a) at (\x1-\dy, \y1+\dx);
    \coordinate (b) at (\x1+\dy, \y1-\dx);
    \coordinate (c) at (\x2+\dy, \y2-\dx);
    \coordinate (d) at (\x2-\dy, \y2+\dx);
    \filldraw[black] (\x1, \y1) circle (2pt) node [anchor=east] {$\mathbf{y}_1$};
    \filldraw[black] (\x2, \y2) circle (2pt) node [anchor=west] {$\mathbf{y}_2$};

    \draw[gray,thick] (a) node [anchor=south] {a} -- 
    (b) node [anchor=north] {b} -- 
    (c) node [anchor=north] {c} -- 
    (d) node [anchor=south] {d} -- (a);
    \draw[black,thick,dashed] (\x1, \y1) -- (\x2, \y2);
  \end{tikzpicture}
  \caption{Single line}
  \label{fig:single-line}
\end{figure}

First, define
%
\begin{align}
  \mathbf{y}_1 &= \mathbf{A} \mathbf{x}_1 + \mathbf{b} \\
  \mathbf{y}_2 &= \mathbf{A} \mathbf{x}_2 + \mathbf{b} \\
\end{align}
%
and
%
\begin{align}
  \bm{\Delta} &= \mathbf{y}_2 - \mathbf{y}_1\\
  l &= \norm{\bm{\Delta}}
\end{align}
%
Next, we define a matrix
%
\begin{align}
  \mathbf{R} = \begin{bmatrix} 0 & -1 \\ 1 & 0 \end{bmatrix}
\end{align}
% 
In practice, applying this to a vector rotates the vector by 90 degrees
counter-clockwise. We now have what we need to define the corners
of the rectangle defining the line.
%
\begin{align}
  \mathbf{a} &= \mathbf{y}_1 + \frac{w}{l} \mathbf{R} \bm{\Delta}\\
  \mathbf{b} &= \mathbf{y}_1 - \frac{w}{l} \mathbf{R} \bm{\Delta}\\
  \mathbf{c} &= \mathbf{y}_2 + \frac{w}{l} \mathbf{R} \bm{\Delta}\\
  \mathbf{d} &= \mathbf{y}_2 - \frac{w}{l} \mathbf{R} \bm{\Delta}\\
\end{align}
%
There is a problem here however. $\mathbf{A}'$ is baked in to several
of the elements. We replace the $\mathbf{y}$s with $\mathbf{x}$s. We
define
%
\begin{align}
  \bm{\Delta}_x = \p{\mathbf{x}_2 - \mathbf{x}_1}
\end{align}
%
which yields
%
\begin{align}
  \bm{\Delta} &= \mathbf{A} \bm{\Delta}_x\\
  l &= \norm{\mathbf{A} \bm{\Delta}_x}
\end{align}
%
We can now rewrite the rectangles as
%
\begin{align}
  \mathbf{a} &= \mathbf{A} \mathbf{x}_1 + \mathbf{b} +
               \frac{w}{l} \mathbf{R} \mathbf{A} \bm{\Delta}_x\\
  %
  \mathbf{b} &= \mathbf{A} \mathbf{x}_1 + \mathbf{b} -
               \frac{w}{l} \mathbf{R} \mathbf{A} \bm{\Delta}_x\\
  %
  \mathbf{c} &= \mathbf{A} \mathbf{x}_2 + \mathbf{b} +
               \frac{w}{l} \mathbf{R} \mathbf{A} \bm{\Delta}_x\\
  %
  \mathbf{d} &= \mathbf{A} \mathbf{x}_2 + \mathbf{b} -
               \frac{w}{l} \mathbf{R} \mathbf{A} \bm{\Delta}_x\\
\end{align}
%
From this, we see that each vertex needs five attributes, its
position, $\bm{\Delta}$, and its offset direction.  Let
$\mathbf{v} \in \mathbb{R}^5$ represent a vertex. The 2D position to
draw each point is
%
\begin{align}
  \mathbf{p} &= \mathbf{A} \begin{bmatrix} v_0 \\ v_1 \end{bmatrix} + \mathbf{b} +
  \frac{v_4 w}{l} \mathbf{R} \mathbf{A} \begin{bmatrix} v_2 \\ v_3 \end{bmatrix}\\
  l &= \norm{\mathbf{A} \begin{bmatrix} v_2 \\ v_3 \end{bmatrix}}
\end{align}
%
where
%
\begin{align}
  \mathbf{a} \quad &=> \quad \mathbf{v} =
            \begin{bmatrix} \mathbf{x}_1^T & \bm{\Delta}^T & 1 \end{bmatrix}^T\\
  %
  \mathbf{b} \quad &=> \quad \mathbf{v} =
            \begin{bmatrix} \mathbf{x}_1^T & \bm{\Delta}^T & -1 \end{bmatrix}^T\\
  %
  \mathbf{c} \quad &=> \quad \mathbf{v} =
            \begin{bmatrix} \mathbf{x}_2^T & \bm{\Delta}^T & 1 \end{bmatrix}^T\\
  %
  \mathbf{d} \quad &=> \quad \mathbf{v} =
            \begin{bmatrix} \mathbf{x}_2^T & \bm{\Delta}^T & -1 \end{bmatrix}^T\\
\end{align}
%
and $\mathbf{A}$, $\mathbf{b}$ and $w$ are passed as uniforms. This way, the
vertices only need to be computed once.

\FloatBarrier
\subsection{Multiple Lines}
When drawing multiple lines, we have to make sure they intersect nicely.
An example is shown in \cref{fig:multi-line}, and there is a gap between
the lines which must be filled in. To do this we draw an additional triangle
between the black dot, corner d and corner e. If the angle between the 
lines went the other way the gap would occur between the black dot, corner
f and corner c. Hence, we draw a triangle here as well.
%
\begin{figure}
  \centering
  \begin{tikzpicture}
    \tikzmath{
      \x1 = 0;
      \y1 = 0;
      \x2 = 4;
      \y2 = 2;
      \x3 = 6;
      \y3 = 0;
      \w = 0.4;
      %
      \l1 = sqrt((\x2-\x1)^2 + (\y2-\y1)^2);
      \dx1 = \w * (\x2 - \x1) / \l1;
      \dy1 = \w * (\y2 - \y1) / \l1;
      %
      \l2 = sqrt((\x3-\x2)^2 + (\y3-\y2)^2);
      \dx2 = \w * (\x3 - \x2) / \l2;
      \dy2 = \w * (\y3 - \y2) / \l2;
    }

    \coordinate (a) at (\x1-\dy1, \y1+\dx1);
    \coordinate (b) at (\x1+\dy1, \y1-\dx1);
    \coordinate (c) at (\x2+\dy1, \y2-\dx1);
    \coordinate (d) at (\x2-\dy1, \y2+\dx1);

    \coordinate (e) at (\x2-\dy2, \y2+\dx2);
    \coordinate (f) at (\x2+\dy2, \y2-\dx2);
    \coordinate (g) at (\x3+\dy2, \y3-\dx2);
    \coordinate (h) at (\x3-\dy2, \y3+\dx2);

    \filldraw[black] (\x1, \y1) circle (2pt);
    \filldraw[black] (\x2, \y2) circle (2pt);
    \filldraw[black] (\x3, \y3) circle (2pt);

    \draw[gray,thick] 
      (a) node [anchor=east] {a} -- 
      (b) node [anchor=north] {b} -- 
      (c) node [anchor=west] {c} -- 
      (d) node [anchor=south] {d} -- 
      (a);
    \draw[gray,thick] 
      (e) node [anchor=south] {e} -- 
      (f) node [anchor=east] {f} -- 
      (g) node [anchor=north] {g} -- 
      (h) node [anchor=west] {h} -- 
      (e);
  \end{tikzpicture}
  \caption{Multiple lines}
  \label{fig:multi-line}
\end{figure}


\end{document}
