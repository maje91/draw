\documentclass[11pt,a4paper]{article}

\usepackage{tikz}
\usepackage[parfill]{parskip}
\usepackage{mathtools}
\usepackage{amsfonts}
\usepackage{placeins}
\usepackage{cleveref}

\begin{document}

\usetikzlibrary{math}

\section{Lines}
\subsection{Single Line}
Let $(x_1, y_1)$ be one side of the line and $(x_2, y_2)$ be the other.
Let $w$ be desired with of the line. \Cref{fig:single-line} shows an
example of a single drawn line. To draw this line on the GPU we need
to pass the points $a$, $b$, $c$, and $d$ to a vertex shader somehow. First,
define
%
\begin{align}
  l &= \sqrt{(x_2 - x_1)^2 + (y_2 - y_1)^2} \\
  \delta x &= (x_2 - x_1) / l \\
  \delta y &= (y_2 - y_1) / l
\end{align}
%
Then
%
\begin{align}
  a &= (x_1 - w \delta y, y_1 + w \delta x) \\
  b &= (x_1 + w \delta y, y_1 - w \delta x) \\
  c &= (x_2 + w \delta y, y_2 - w \delta x) \\
  d &= (x_2 - w \delta y, y_2 + w \delta x)
\end{align}
%
From this, we see that each vertex needs four attributes, its position and its offset.
Let $\mathbf{v} \in \mathbb{R}^4$ represent a vertex. The 2D position to draw each point
should be
%
\begin{align}
  \mathbf{p} = \begin{bmatrix}
    v_0 + w v_2 \\
    v_1 + w v_3
  \end{bmatrix}
\end{align}
%
where
%
\begin{align}
  a \quad &=> \quad \mathbf{v} = \begin{bmatrix} x_1 & y_1 & -\delta y & \delta x \end{bmatrix}^T\\
  b \quad &=> \quad \mathbf{v} = \begin{bmatrix} x_1 & y_1 & \delta y & -\delta x \end{bmatrix}^T\\
  c \quad &=> \quad \mathbf{v} = \begin{bmatrix} x_2 & y_2 & \delta y & -\delta x \end{bmatrix}^T\\
  d \quad &=> \quad \mathbf{v} = \begin{bmatrix} x_2 & y_2 & -\delta y & \delta x \end{bmatrix}^T
\end{align}
%
and $w$ is passed as a uniform. This way, the vertices only need to be computed once.
%
\begin{figure}
  \centering
  \begin{tikzpicture}
    \tikzmath{
      \x1 = 0;
      \y1 = 0;
      \x2 = 4;
      \y2 = 2;
      \l = sqrt((\x2-\x1)^2 + (\y2-\y1)^2);
      \w = 0.4;
      \dx = \w * (\x2 - \x1) / \l;
      \dy = \w * (\y2 - \y1) / \l;
    }

    \coordinate (a) at (\x1-\dy, \y1+\dx);
    \coordinate (b) at (\x1+\dy, \y1-\dx);
    \coordinate (c) at (\x2+\dy, \y2-\dx);
    \coordinate (d) at (\x2-\dy, \y2+\dx);
    \filldraw[black] (\x1, \y1) circle (2pt) node [anchor=east] {$(x_1, y_1)$};
    \filldraw[black] (\x2, \y2) circle (2pt) node [anchor=west] {$(x_2, y_2)$};

    \draw[gray,thick] (a) node [anchor=south] {a} -- 
    (b) node [anchor=north] {b} -- 
    (c) node [anchor=north] {c} -- 
    (d) node [anchor=south] {d} -- (a);
    \draw[black,thick,dashed] (\x1, \y1) -- (\x2, \y2);
  \end{tikzpicture}
  \caption{Single line}
  \label{fig:single-line}
\end{figure}

\FloatBarrier
\subsection{Multiple Lines}
When drawing multiple lines, we have to make sure they intersect nicely.
An example is shown in \cref{fig:multi-line}, and there is a gap between
the lines which must be filled in. To do this we draw an additional triangle
between the black dot, corner d and corner e. If the angle between the 
lines went the other way the gap would occur between the black dot, corner
f and corner c. Hence, we draw a triangle here as well.
%
\begin{figure}
  \centering
  \begin{tikzpicture}
    \tikzmath{
      \x1 = 0;
      \y1 = 0;
      \x2 = 4;
      \y2 = 2;
      \x3 = 6;
      \y3 = 0;
      \w = 0.4;
      %
      \l1 = sqrt((\x2-\x1)^2 + (\y2-\y1)^2);
      \dx1 = \w * (\x2 - \x1) / \l1;
      \dy1 = \w * (\y2 - \y1) / \l1;
      %
      \l2 = sqrt((\x3-\x2)^2 + (\y3-\y2)^2);
      \dx2 = \w * (\x3 - \x2) / \l2;
      \dy2 = \w * (\y3 - \y2) / \l2;
    }

    \coordinate (a) at (\x1-\dy1, \y1+\dx1);
    \coordinate (b) at (\x1+\dy1, \y1-\dx1);
    \coordinate (c) at (\x2+\dy1, \y2-\dx1);
    \coordinate (d) at (\x2-\dy1, \y2+\dx1);

    \coordinate (e) at (\x2-\dy2, \y2+\dx2);
    \coordinate (f) at (\x2+\dy2, \y2-\dx2);
    \coordinate (g) at (\x3+\dy2, \y3-\dx2);
    \coordinate (h) at (\x3-\dy2, \y3+\dx2);

    \filldraw[black] (\x1, \y1) circle (2pt);
    \filldraw[black] (\x2, \y2) circle (2pt);
    \filldraw[black] (\x3, \y3) circle (2pt);

    \draw[gray,thick] 
      (a) node [anchor=east] {a} -- 
      (b) node [anchor=north] {b} -- 
      (c) node [anchor=west] {c} -- 
      (d) node [anchor=south] {d} -- 
      (a);
    \draw[gray,thick] 
      (e) node [anchor=south] {e} -- 
      (f) node [anchor=east] {f} -- 
      (g) node [anchor=north] {g} -- 
      (h) node [anchor=west] {h} -- 
      (e);
  \end{tikzpicture}
  \caption{Multiple lines}
  \label{fig:multi-line}
\end{figure}

\FloatBarrier
\subsection{Stretching}
When zooming in and out of lines, we might want the line width to remain constant.
This is typically the case for plots. Define the matrix
%
\begin{align}
  \mathbf{Z} = \begin{bmatrix}
    z_x & 0 \\
    0 & z_y \\
  \end{bmatrix}
\end{align}
%
where $z_x$ defines the zoom in the x-direction, and $z_y$ in the y-direction.
We apply the zooming to $\mathbf{p}$ to get
%
\begin{align}
  \mathbf{Zp} = \begin{bmatrix}
    z_x v_0 + z_x w v_2 \\
    z_y v_1 + z_y w v_3
  \end{bmatrix}
\end{align}
%
which we see affects the width of the line. To counteract this we modify
$\mathbf{p}$ to be
%
\begin{align}
  \mathbf{p}_s = \begin{bmatrix}
    v_0 + \frac{w v_2}{z_x} \\
    v_1 + \frac{w v_3}{z_y}
  \end{bmatrix}
\end{align}
%


\end{document}
